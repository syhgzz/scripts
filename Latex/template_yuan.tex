% Latex课程报告模板 袁老师封面版
 
\documentclass{article}
\usepackage[UTF8,zihao=-4]{ctex}
\usepackage{amsfonts} 
\usepackage{amsmath} 
\usepackage[top=2.54cm, bottom=2.54cm, left=3.18cm, right=3.18cm]{geometry}
\geometry{paper=a4paper}
\usepackage{graphicx}
\usepackage{gbt7714}
\usepackage{setspace}
\usepackage{indentfirst}
\usepackage{titlesec}
\usepackage{fancyhdr}  
\usepackage{listings} % 引入 listings 宏包
\usepackage{xcolor}   % 支持颜色  



%%%%%%%%%%%%% 自定义设定 %%%%%%%%%%%%


% 自定义字体
\newCJKfontfamily\heititext{SimHei}[
  Path = C:/Windows/Fonts/,  % Windows字体目录
  AutoFakeBold = 2,        % 加粗强度(3.5为最佳视觉效果)
]


\newCJKfontfamily\simsuntext{SimSun}[
  Path = C:/Windows/Fonts/,  % Windows字体目录
  AutoFakeBold = 2,        % 加粗强度(3.5为最佳视觉效果)
]

\newCJKfontfamily\hwtext{STXinwei}[
  Path = C:/Windows/Fonts/,  % Windows字体目录
  AutoFakeBold = 2,        % 加粗强度(3.5为最佳视觉效果)
]

% 以下是MacOS自定义字体
% \newCJKfontfamily\heititext{SimHei}[
%   Path = /Users/zhouzhuo/Library/Fonts/,  % Windows字体目录
%   AutoFakeBold = 2,        % 加粗强度(3.5为最佳视觉效果)
% ]

% \newCJKfontfamily\simsuntext{SimSun}[
%   Path = /Users/zhouzhuo/Library/Fonts/,  % Windows字体目录
%   AutoFakeBold = 2,        % 加粗强度(3.5为最佳视觉效果)
% ]

% \newCJKfontfamily\hwtext{simkai}[
%   Path = /Users/zhouzhuo/Library/Fonts/,  % Windows字体目录
%   AutoFakeBold = 2,        % 加粗强度(3.5为最佳视觉效果)
% ]


% 自定义命令
% 三角等号
\newcommand{\triangleq}{\stackrel{\mathrm{def}}{=}} 
% 自定义变量
\newcommand{\TITLE}{人工智能于安防领域案例:蒙面恐怖分子检测} %标题
\newcommand{\AUTHOR}{周卓} %作者
\newcommand{\STUDENTNO}{2240201012} %学号
\newcommand{\sizethirty}{\fontsize{30pt}{50pt}}  % 30号字(初号)

% 行距与缩进
\onehalfspacing
\setlength{\parindent}{2em}

% 自定义各级标题格式字体
\titleformat{\section}
  {\heititext \zihao{3} \bfseries \centering}
  {\thesection}
  {1em}{}
\titlespacing{\section}{0pt}{24pt}{18pt}

\titleformat{\subsection}
  {\simsuntext \zihao{4} \bfseries \raggedright}
  {\thesubsection}
  {1em}{}
\titlespacing{\subsection}{0pt}{18pt}{12pt}

\titleformat{\subsubsection}
  {\simsuntext \zihao{-4} \bfseries \raggedright}
  {\thesubsubsection}
  {1em}{}
\titlespacing{\subsubsection}{0pt}{12pt}{6pt}

% 代码块样式定义
\lstset{
    basicstyle=\ttfamily\small, % 基本字体样式
    keywordstyle=\color{blue},  % 关键字颜色
    commentstyle=\color{green}, % 注释颜色
    stringstyle=\color{red},    % 字符串颜色
    numbers=left,               % 显示行号
    numberstyle=\tiny\color{gray}, % 行号样式
    frame=single,               % 边框样式
    breaklines=true,            % 自动换行
    tabsize=4                   % 制表符宽度
}


%%%%%%%%%%%%% 文档开始 %%%%%%%%%%%%

\begin{document}


%%%%%%%%%%%%% 封面页 %%%%%%%%%%%%

\pagestyle{empty}
\begin{flushright}
{
     {\simsuntext \zihao{-4} \bfseries 2024-2025学年第二学期}
    
}
\end{flushright}


\begin{figure}[h]
    \centering
    \includegraphics[width=0.8\textwidth]{./城院logo3.png}
    % \caption{}
\end{figure}
% \vspace*{12em}
\begin{center}
    {\hwtext\bfseries\sizethirty 
    《机器学习》\\[1.5\baselineskip]  % 手动调整行距
    课程报告}
\end{center}
\begin{figure}[h]
    \centering
    \includegraphics[width=0.33\textwidth]{./城院logo2.png}
    % \caption{}
\end{figure}
% \vspace{3em}

\newcommand{\fssi}{\fangsong\zihao{4}}

\begin{center}
{ \fssi % 这里的字号也可以用别的方式修改

\makebox[4em][s]{姓名}:\hspace{1em}\underline{\makebox[14em][c]{\AUTHOR}}\\
\vspace{1em}
\makebox[4em][s]{学号}:\hspace{1em}\underline{\makebox[14em][c]{\STUDENTNO}}\\
\vspace{1em}
\makebox[4em][s]{专业班级}:\hspace{1em}\underline{\makebox[14em][c]{电子信息2402}}\\
\vspace{1em}
\makebox[4em][s]{所在学院}:\hspace{1em}\underline{\makebox[14em][c]{信息与电气工程学院}}\\
\vspace{1em}
\makebox[4em][s]{指导老师}:\hspace{1em}\underline{\makebox[14em][c]{袁建涛/高峰}}\\
\vspace{1em}
\makebox[4em][s]{日期}:\hspace{1em}\underline{\makebox[14em][c]{2025.4.16}}\\
}
\end{center}



%%%%%%%%%%%%% 标题 摘要 %%%%%%%%%%%%



\title{\TITLE}
% \author{\AUTHOR~ \STUDENTNO}
\date{}
\maketitle

\thispagestyle{empty}

\begin{center}
  {\heititext\zihao{3}\bfseries 摘要}  % 标题加粗
\end{center}
\vspace{0.5em}
{
    \simsuntext\zihao{-4}\linespread{1.5}\selectfont
    \hspace{2em}本研究针
    \par  % 确保行距生效
    \vspace{0.5em}
    \noindent{\simsuntext\zihao{-4}\bfseries 关键词:} \simsuntext\zihao{-4}智能安防;目标识别;人脸检测;
}

%%%%%%%%%%%%% 目录 %%%%%%%%%%%%

\newpage
\pagestyle{plain}
\pagenumbering{roman}
\tableofcontents

%%%%%%%%%%%%% 正文 %%%%%%%%%%%%

\newpage
\pagestyle{fancy}
\fancyhf{}               % 清空默认页眉页脚
\renewcommand{\headrulewidth}{0pt} % 取消页眉线
\pagenumbering{arabic} \setcounter{page}{1} % 重置为阿拉伯数字
\cfoot{第 \thepage\ 页 \hspace{0.5em} 共 \pageref{EndBody} 页}

\section{引言}
  随    
\section{技术基础}
  \subsection{算法综述}
    本研究\cite{mishra2022drone}


  
% 尾页标签, 用于标记页数
\label{EndBody}        


%%%%%%%%%%%%%  参考文献 %%%%%%%%%%%%
\newpage
\pagestyle{empty}

\bibliographystyle{gbt7714-numerical}
\bibliography{xinhaochuli.bib}


%%%%%%%%%%%%% 文档结束 %%%%%%%%%%%%
\end{document}